%************************************************
\chapter{State of the Art}\label{ch:sota}
%************************************************
When talking about cognitive sciences, several studies exist that illustrate the underpinnings of human perception and cognition of basic dimensions of music, such as loudness, pitch, rhythm and timbre (e.g., see \cite{justus2002music}). Further research has focused on higher-level concepts related to music, including the perception of its emotive content and the way in which we tend to express it (\cite{juslin2004expression}), as well as performance specific traits (\cite{palmer1997music}). Recently, in the attempt of understanding even more thoroughly the nature of our appreciation of beauty, a new field of research, known as neuroaesthetics, has started to investigate the correlation between empirical aesthetics and cognitive neuroscience (\cite{pearce2016neuroaesthetics}).

As it often happens, the first studies in the field have focused on the visual domain. \cite{kawabata2004neural}, for example, observed that some areas of the brain get activated in different ways when subjects are shown paintings labeled as ``beautiful'', as opposed to ``neutral'' or ``ugly''.

Focusing on music, the work of \cite{brattico2013neuroaesthetics} presents a good analysis of the current state of the research. In their conclusions, it’s acknowledged the fact that neuroaesthetics of music is still a field in its infancy, and that more empirical research is needed in order to clarify its effectiveness, as well as the practical scenarios where such knowledge could be useful for. The most relevant aspect for the present work, however, is the fact that they draw from psychological research to restate the main factors contributing to an aesthetic experience: the characteristics of the listener, of the listening situation, and, of course, of the music itself. While it is known that all of them have an important role in defining the aesthetic experience of music (e.g., see \cite{hargreaves201021}), it’s still not clear their reciprocal influence, nor their relative combination. 

Neuroaesthetics provide us with useful information regarding our neurological reactions when we witness acts of beauty. If it is true that specific brain activity is observed in these situations, not so much we can say about whether these activities are caused by specific properties of the musical object. Following the suggestion from \citeauthor{brattico2013neuroaesthetics}, we thus aim at developing an empirical model that we hope will shed some light on the matter. To do so, we will take advantage of the recent advances of AI, in particular of those related to music information retrieval and computational aesthetics.

Once again, our approach is supported by a wealth of research in the domain of the visual arts, where several objective key properties seem to have been observed to be present in beautiful images. \cite{jacobs2016aesthetics}, for example, found that some of these properties might correspond to lower spatial frequencies, oblique orientations, higher intensity variation, higher saturation, and overall redness. \cite{schifanella2015image} developed a model which was able, although with some limitations, to surface beautiful but unpopular pictures from a pool of items uploaded to the photo-sharing platform Flickr. Their approach is based on computing specific descriptors related either to color (e.g., contrast, hue, saturation), spatial arrangement (e.g., symmetry, rule of thirds), or texture (e.g., entropy, energy, homogeneity), and comparing them against the same features computed from a ground-truth of crowdsourced pictures deemed to be beautiful.

Some theories that try to quantify beauty in music, or at least to give some related measure, have already been proposed. \cite{manaris2002progress}, and \cite{manaris2005zipf}, for example, conducted experiments exploiting a statistical technique known as Zipf’s law on a corpus of MIDI-encoded pieces, suggesting that this technique might be used as a metric for aesthetic evaluation. The music pieces used in their experiments were reportedly selected ``by a member [...] with an extensive music theory background'', are all pieces belonging to the classical music genre (as much as the vagueness of this label implies), and have been cut down to two minutes, to prevent fatigue in the listeners. We believe that these choices could introduce a strong bias to the experiment, since many assumptions are implicitly made here, or not explicitly discarded.

\cite{hudson2011musical} advances an hypothesis that roots in information theory, proposing that compressibility and music appreciation are strictly bound. More specifically, the cognitive process of finding patterns more or less hidden inside a piece of music directly relates to a reward system responsible for our appreciation of it. This hypothesis, although fascinating, lacks the support of empirical experiments, and should therefore be taken into consideration with care. A related study by \cite{mcdermott2013summary} shows that the auditory system tends to summarize temporal details of sound textures using time-averaged statistics, especially when the length of the sound is moderate to high.

\cite{brattico2017global}, on the same line, and drawing from the studies in visual aesthetics, put forward the hypothesis that our auditory system extracts global features from musical stimuli, and then passes them to the high-level processing responsible for the outcomes of the musical experience, including aesthetic judgment. These global features, analogously to visual features, are defined in terms of distribution of spectral energy, musical texture, expressivity, tempo and mode, and more.

Moreover, they propose that the creation of musical beauty is not limited to any particular style, method, genre, or form, and therefore implying that the aforementioned model could be applied to any music piece. In particular, this last study shows appeal, mainly due to the existing background in visual processing. We thus question if, starting from a ground-truth of crowdsourced music pieces deemed to be beautiful, as proposed by \citeauthor{schifanella2015image} in the visual domain, it is possible to extract global features that somehow show similar properties along the various pieces of music, and to use them to accordingly classify other pieces of music, regardless of their genre or style.

%*****************************************
%*****************************************
%*****************************************
%*****************************************
%*****************************************
