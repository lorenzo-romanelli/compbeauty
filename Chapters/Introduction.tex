\chapter{Introduction}\label{ch:intro}
<<Beauty lies in the eyes of the beholder>>. <<De gustibus non est disputandum>>.

Everyone has heard these sayings. But are they true? And when I say true I mean: is it really how we behave?

From my personal experience, I do know that when I am witnessing some act of beauty -- be it listening to a song, or standing in front of a breathtaking landscape -- one of my first thoughts is often something along the lines of: <<I wish everyone else could experience the same>>. (Which is usually followed by me posting that song on Facebook. Or a picture of that landscape on Instagram. But this is probably to be attributed to my social media addiction.)

My point is that, despite all, people have been sharing their views about beauty, and have been doing so since the dawn of mankind. Plato; the sophists; Hegel, Kant and Schopenhauer; they all tried to disentangle our complex relationship with aesthetic judgments. More recently, even psychologists, neuroscientists and computer scientists joined this quest, each of them trying to provide an explanation on the matter from their own field's perspective.

It is true: no universal definition yet exists for beauty. But still, if aesthetic experiences can only be personal, if any discussion over beauty is futile, why would we still bother engaging in discussions that, in the end, we know will lead nowhere? When I go to a friend and say: <<Hey, listen to this song. It is beautiful, isn't it?>>, I expect him or her to agree with me. And, admittedly, I can get angry if he or she doesn't. That is because in my head I can associate the experience of finding something beautiful to characteristics of the song, or to the situation I find myself in, or even to a mix of both, that I believe can be universal, and recognizable by others, even if just because of sympathy.

This process of self-thought is what stimulated the work presented in this thesis. When I say that a piece of music is beautiful, I do provide reasons why I think so. Think about reviewers, whose job is to praise or criticize works of art. You would not trust a critic's opinion if he did not adduce reasons over why something can or cannot be considered beautiful. I feel like this is a considerable gap in the study of beauty: an approach grounded in the actual features of the object that is being judged is needed. I believe this can be achieved by looking at the sources in which people talk, or write, about beauty and music.

The Internet nowadays is an incredible source of information easily and readily available for most users to consult and use. Among this sea of data, online music magazines flourish. And the tools for extracting meaningful insights froms such a vast amount of data are there, thanks to the impressive developments in artificial intelligence we are witnessing these days. Someone just needs to take the plunge and start diving into an analysis of what people refer to when talking about musical beauty.

\section{Organization}
The content of the thesis will be organized as follows.

\autoref{ch:sota} reports the current state of the art related to the study of beauty from the standpoint of different disciplines, including philosophy, cognitive neuroscience, and computer science. Our research question is going to be framed more precisely here, along with some of the limitations we have to take into account.

\autoref{ch:methodology} deals with our approach to answering such question. The dataset, the techniques and the steps adopted will be described in detail with a proper mathematical language.

\autoref{ch:results} contains the results obtained by applying our proposed methodology on the chosen dataset. A parallel experiment conducted on a different dataset will also be briefly described.

In \autoref{ch:conclusion}, finally, we further discuss our results while drawing some conclusions, as well as outlining guidelines and ideas for further work.
