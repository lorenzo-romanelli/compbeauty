%************************************************
\chapter{State of the Art}\label{ch:sota}
%************************************************
When talking about cognitive sciences, several studies exist that illustrate the underpinnings of human perception and cognition of basic dimensions of sound, such as loudness, pitch, rhythm and timbre (e.g., see \cite{justus2002music}). Further research has focused on higher-level concepts related to music, including the perception of its emotive content and the way in which we tend to express it (\cite{juslin2004expression}), as well as performance specific traits (\cite{palmer1997music}).

In the Music Information Retrieval (\acsfont{MIR}) field\footnote{Music Information Retrieval, as the name suggest, is the interdisciplinary science dealing with the study of techniques aimed at extracting information from music sources in an automatic way.}, it is useful to categorize these musical dimensions, most commonly referred to as \emph{descriptors}, using a hierarchy organized in three levels of abstraction (\cite{gouyon2008content}, among others). Climbing the ladder of such hierarchy up to the top means starting from the most fundamental acoustic features, to be extracted directly from the signal, and progressively building on top of them to get to model more complex concepts derived from music theory, or musicology, or even from cognitive and social phenomena.

The organization of the three levels of this hierarchy, in order of increasing complexity of the features associated with each level, is defined as follows:

\begin{enumerate}
	\item \emph{low-level descriptors} -- loudness, pitch, timbre, onsets, \ldots
	\item \emph{mid-level descriptors} -- tempo, tonality, modality, \ldots
	\item \emph{high-level descriptors} -- genre, mood, instrumentation, \ldots
\end{enumerate}

High-level descriptors are referred to also as \emph{semantic} descriptors, for they require an additional induction from users. In other words, we cannot rely solely on data computed directly on an audio signal\footnote{Nor from the symbolic information (usually the score) associated with a piece of music.} to define concepts such as the mood of a song. We in fact need to first give an interpretation of terms like \emph{happiness} or \emph{sadness} from the user's perspective, contextualize them, and then approach the study of how these interpretations relate with low or mid-level descriptors extracted from the music. Models used for high-level descriptors have to rely on prior knowledge which is always more or less biased towards the end users of the specific application.

Suppose you had to design an algorithm for a music recommender system based on genre similarity. How would you define which genres are similar to each other? The metrics suited for the task can be many\footnote{This is just an example built on common sense; the topic is huge, research on it is abundant and beyond the scope of this work.}: instrumentation, tempo, rhythm, most likely a mix of them and more, or even data which is not necessarily bound to the audio information itself (I am talking about \emph{metadata}). The opinion of a domain expert, even if technically more correct or informed, might be less suited for this application than the perspective of the layman using the platform everyday, who doesn't care if two pieces of music are both from Detroit-based producers.

This problem bound to interpretation of high level semantics is also known as the \emph{semantic gap}. To put it in the words of \citeauthor{smeulders2000content}, the semantic gap is <<[...] The lack of coincidence between the information that one can extract from the (sensory) data and the interpretation that the same data has for a
user in a given situation>> (\cite{smeulders2000content}). The semantic gap issue becomes even more relevant as the concept we are trying to model becomes more abstract; this is indeed the context where the problem I am going to outline in the next paragraphs finds its place and some of its practical justifications.

Given the promising advances seen in the field in the last fifteen years, it is surprising to see how the study of the concept of musical beauty from an \acsfont{MIR} perspective is barely considered. Dealing with beauty -- not only when talking about music -- is of course a tricky undertaking. Everyone has heard the common saying that <<beauty lies in the eye of the beholder>>\footnote{The quote as we know it appeared for the first time in the book \citetitle{hungerford1878molly} by Margaret Wolfe Hungerford  (\cite{hungerford1878molly}).}, which, perhaps less poetically, suggests that the experience of the beautiful can only be interpreted as a subjective phenomenon detached from any objective feature of what caused it.

If it was true that we can't explain beauty other than by accepting its independence from any formal, observable and quantifiable property, then transposing the task in an \acsfont{MIR} context would have no purpose, and we'd better abandon our hopes. Fortunately, there exists a wealth of research suggesting that a different point of view on the matter might make more sense -- not just from a philosophical perspective. Next to aesthetic theories, studies in cognitive neurosciences and artificial intelligence add value to the hypothesis that aesthetic experiences can be explained at least in part by objective foundations.

What follows is a discussion on those pieces of research that I consider relevant for the work of the present thesis, as well as for giving an outline of its limitations; as such, I don't expect it to be taken as an exhaustive literature review on every possible theory about the nature of art and beauty, and even less on the philosophy of aesthetics as a whole\footnote{For the reader interested in a more comprehensive introduction to Asthetics, I'd suggest to head to other resources; \cite{graham2005philosophy} and \cite{tatarkiewicz2006history} are good starting points, which I myself will address multiple times during my discussion.}.

\section{What is beauty? A review of aesthetic theories}\label{sec:aesthetics}
Any discourse about beauty must deal with the fact that there isn't a consensus on its nature. This question has been debated for at least 2\,500 years and has been given a wide variety of answers. Immanuel Kant thought one premise of beauty was an attitude of ``disinterested contemplation'' (\cite{kant2001critique}), whereas Friedrich Nietzsche dismissed this notion and underlined the impact of sensual attraction (\cite{nietzsche1998genealogy})\footnote{There is a whole current of thought, known as \emph{Darwinian aesthetics} or \emph{evolutionary aesthetics}, suggesting that humans may be biologically primed to find particular features more beautiful, because these features may have been selected for optimal survival (e.g., \cite{thornhill1998darwinian}, \cite{grammer2003darwinian}), which will not be addressed here.}. For the poet John Keats, beauty equaled truth (\cite{keats1898ode}), while Stendhal, the French novelist, characterized beauty as the ``promise of happiness'' (\cite{stendhal1927love}). Each of these theories is respected; not one is universally accepted.

In my discussion, I will adopt the \emph{\citelist{beautyoxford}{publisher}} definition of beauty as a starting point\footnote{See \cite{beautyoxford}.}:
\begin{quote}
A combination of qualities, such as shape, colour, or form, that pleases the aesthetic senses, especially the sight.
\end{quote}

I don't particularly like this definition, for two reasons. First, there is an explicit reference to its ``objective'' interpretation, as the term gets bound to concrete qualities and ignores any possible subjective implication. Moreover, this definition suggests that the sight somehow holds a privileged position among the aesthetic senses -- whichev\-er those senses are. Does it mean that things that please the eye are to be considered more beautiful than, say, music? Or maybe that we perceive beauty better through sight? Are there more beautiful objects in the visual domain than in others? How can we even quantify beauty\footnote{My interpretation is that, generally speaking, sight is seen -- forgive the wordplay -- as the most developed of the human senses. Here, this diffused opinion introduces a bias, perhaps to help contextualizing such a broad topic in the limited space allowed by a dictionary.}?

\subsection{Beauty as aesthetic pleasure}\label{subsec:pleasure}
However, not everything should be thrown away. The definition in fact mentions one aspect that is commonly addressed in the philosophical discourse on beauty: beautiful objects cause pleasure to -- I would rather say \emph{through} -- the aesthetic senses (e.g. \cite{tatarkiewicz2006history}). It is a distinctive kind of pleasure, which exists in a different manner than from the pleasures deriving from a good meal, or fresh air, or a good bath (\cite{ingarden1964artistic}). For example, the immediate pleasure arising from having a cold drink on a hot day lies exclusively in a positive sensation of the body and has little to do with aesthetic appreciation of the object. In contrast, perceivers look at a painting not to please their body, but to enjoy the painting's beauty (\cite{reber2004processing}). As such, this peculiar type of pleasure is usually referred to as \emph{aesthetic pleasure} (\cite{graham2005philosophy}).

It has been observed from ancient times that it seems contradictory to describe something as beautiful and deny that we are in any way pleasurably affected by it. As \citeauthor{graham2005philosophy} exemplifies, the same thing cannot be said for other concepts such as colours. People usually prefer one colour to another; they can even be said to have a favourite colour, but we could not tell that just by looking at their use of colour words alone. Describing an apple as a <<red apple>> doesn't imply that I favour red apples over green apples, whereas if I say <<a \emph{beautiful} red apple>>, you immediately get that I am attributing a positive value to that apple\footnote{And the contrary can be said when using the word \emph{ugly}.}.

This said, there are important questions arising from the previous observation: can we identify some kind of connection between purely descriptive terms (such as \emph{red} or \emph{green}) and the evaluative term \emph{beautiful}? If so, where does this connection lie? The tradition in Aesthetics tells us that usual answers to these questions fall into one of three currents of thought. I have already hinted at some of them, but let's try to describe the overall picture in a bit more detail.

\subsection{Subjectivism, objectivism, interactionism}\label{subsec:subj-obj-inter}
The philosopher David Hume is probably the most renowned exponent of the so-called \emph{subjectivist} view, a view which anyways dates back at least to the Sophists (\cite{tatarkiewicz2006history}). It is here that sayings such as <<Beauty lies in the eyes of the beholder>> and <<De gustibus non est disputandum\footnote{Which roughly translates into <<Taste cannot be debated>>.}>> would find their place. Subjectivists state that beauty is a function of idiosyncratic qualities of the perceiver; which -- coming back to the example of colours -- is to say that terms like red and green identify real properties of the apple, where instead the term beautiful says something about the person who uses it. This perspective, of course, implies that all efforts to identify the laws of beauty would be futile:
\begin{quote}
<<To seek the real beauty, or the real deformity, is as fruitless an enquiry, as to seek the real sweet or real bitter.>>

(\cite{hume1757standard})
\end{quote}

On the opposite, the \emph{objectivist} position sees beauty as a property of an object that produces a pleasurable experience in any suitable perceiver (\cite{tatarkiewicz2006history}). Eduard Hanslick, one of the most respected music critics of the 19th century, states in his foundational book \emph{The Beautiful in Music} that <<[...] Although the beautiful exists for the gratification of an observer, it is independent of him (\cite{hanslick1957beautiful})>>. This perspective finds one of its earliest theorists as far as Plato; it was incredibly popular in the 16th century, to the extent that artists started introducing books of patterns that other artists could combine with each other in order to create beauty (\cite{gombrich1995story}); and it inspired a great deal of psychological research in the 20th century in the attempt of identifying the critical contributors to beauty (e.g., see \cite{birkhoff1933aesthetic}, \cite{arnheim1974art}, \cite{gombrich1980sense}, \cite{gombrich1995story}, \ldots).

Between subjectivists and objectivists we can identify a third current of thought, known as \emph{interactionism}. It tends to be the view adopt\-ed in most modern philosophical -- and not -- analyses. What this theory states is that the sense of beauty emerges from patterns in the way people and objects relate (e.g., see \cite{merleau1964primacy} and \cite{ingarden1985selected}). Put this way, it is no surprise that interactionism is a favourite among cognitive neuroscientists approaching the study of beauty -- this is a relatively young field called \emph{neuroaesthetics} -- as it suggests a discrete neural basis (\cite{conway2013neuroaesthetics}). I will come back to this point later.

\cite{graham2005philosophy} reports an interesting argument against pure subjectivism, which I will describe in \autoref{sec:question} and to which my research question will be closely related. \citeauthor{graham2005philosophy}'s point\footnote{To be fair, his argument seems to be a favourite among those who discard subjectivism, but is not clear who was the first person to bring it forward (probably Thomas Reid, a contemporary of David Hume).} finds its roots in the theory of aesthetic judgments proposed by Immanuel Kant in the \citetitle{kant2001critique}, first published in 1790. For this reason, in the next section I am going to briefly outline Kant's idea about what kind of judgment is it that results in our saying that something is beautiful.

\subsection{Kant's aesthetics}\label{subsec:kant}
According to Kant, aesthetic judgments are identified by four distinguishing features. First, they must be \emph{disinterested}: we take pleasure in something because we judge it beautiful, rather than judging it beautiful because we find it pleasurable. The latter type of judgment would be more like a judgment of the \emph{agreeable}, as when we say <<I like the taste of avocado>>.

Aesthetic judgments, in Kant's view, are also both \emph{universal} and \emph{necessary}. This means that the activity of such judgment involves the instrinsic expectation from others to agree with us. We may say that <<Beauty is in the eye of the beholder>>: but that is not how we act. If I say <<I like the taste of avocado>>, whereas you do not, I can't give you reasons to like the taste of avocado; you just don't. But we do debate about our aesthetic judgements -- especially about works of art. What's more, we tend to believe that such debates and arguments can actually achieve something. For many purposes, beauty behaves as if it was a real property of an object, like its weight or chemical composition. But Kant insists that universality and necessity are in fact a product of the human mind\footnote{This is a similar view to what interactionists propose.}, in a process that Kant calls \emph{common sense}. The consequence, of course, is that there is no objective property of a thing that makes it beautiful.

Finally, through aesthetic judgments beautiful objects appear to be ``purposive without purpose''. An object's purpose is the concept according to which it was made, such as a table in the mind of the carpenter. An object is \emph{purposive} if it appears to have such a purpose, or if, in other words, it appears to have been made or designed. It is part of the experience of beautiful objects, Kant argues, that they should affect us as if they had a purpose, although no particular purpose can be found (\cite{kant2001critique}).

\section{Neuroaesthetics}\label{sec:neuroaesthetics}
Recently, in the attempt of understanding even more thoroughly the nature of our appreciation of beauty, a new field of research, known as neuroaesthetics, has started to investigate the correlation between empirical aesthetics and cognitive neuroscience (\cite{pearce2016neuroaesthetics}). Neuroaestheticians adopt a more grounded approach to the study of beauty than philosophers, in that the former seek to observe recurrent patterns in neurological reactions when the perceiver witnesses acts of beauty. This said, we should not make the mistake of thinking that neuroaesthetics and traditional aesthetics are two completely disjoint fields. I  already mentioned in \autoref{subsec:subj-obj-inter} how the interactionist perspective is a favourite among neuroaestheticians (e.g., \cite{juslin2013everyday} and \cite{reber2004processing} are two pieces of research where the authors explicitly take the interactionist side). The influence of Kant's thought appears to be quite dominant as well (\cite{conway2013neuroaesthetics}).

As it often happens, the first studies in the field have focused on the visual domain. In \cite{kawabata2004neural}, for example, subjects were shown paintings previously classified by the subjects themselves as ``beautiful'', as opposed to ``neutral'' or ``ugly''. By using a technique known as f\acsfont{MRI} (\emph{functional Magnetic Resonance Imaging}), \citeauthor{kawabata2004neural} observed that the perception of different categories of paintings are associated with distinct and specialized visual areas of the brain, that the orbitofrontal cortex is differentially engaged during the perception of beautiful versus ugly stimuli, regardless of the category of painting, and that the perception of stimuli as beautiful or ugly mobilizes the motor cortex differentially. 

\subsection{Neuroaesthetics of music}\label{subsec:music-neuroaesthetics}
Focusing on music, the work of \cite{brattico2013neuroaesthetics} presents a good analysis of the current state of the research. Several neuroimaging studies of musical listening confirm the role of the orbitofrontal cortex in positive affective experiences associated with aesthetic judgments of preference or beauty for music (e.g., see \cite{alluri2012large}, \cite{brattico2011functional}, and \cite{blood2001intensely}\footnote{\citeauthor{blood2001intensely} also highlight how pleasure tends to accompain experiences of beauty, providing an empirical motivation to what has been discussed in \autoref{subsec:pleasure}.}), as it was observed for paintings.

\citeauthor{brattico2013neuroaesthetics} argue that there is one important, distinctive difference between neuroaesthetics of art in general (i.e., of visual arts) and neuroasthetics of music, in that the subject of the latter is a complex multidimensional, auditory signal extended in time and processed in distinct neural pathways from visual stimuli. One consequence of this distinction lies in the specific focus that must be called for in a neuroaesthetic of music on the role of time: a piece of music cannot be viewed as a static entity, but rather one that unfolds in time, generating and manipulating expectations\footnote{A framework for linking expectations based on statistical learning to aesthetic responses has been proposed in \cite{huron2006sweet}. According to \citeauthor{huron2006sweet}, an event that is unexpected but ultimately innocuous is capable of inducing a negative prediction response that increases, in a process called \emph{contrastive valence}, the relatively positive limbic effect of the subsequent reaction or appraisal responses. Empirical evidence supports the theory that confirmation or violation of expectations is capable of leading to
aesthetic experiences (e.g., \cite{vitz1966affect}, \cite{crozier1974verbal}).} and interpretations in order to induce an aesthetic experience.

In \citeauthor{brattico2013neuroaesthetics}'s conclusions, it is acknowledged the fact that neuroaesthetics of music is still a field in its infancy, and that more empirical research is needed in order to clarify its effectiveness, as well as the practical scenarios where such knowledge could be useful for. They also draw from psychological research to restate the three main factors contributing to an aesthetic experience: the characteristics of the listener, of the listening situation, and, of course, of the music itself. While it is known that all of them assume an important role in defining the aesthetic experience of music (e.g., see \cite{hargreaves201021}), it still is not clear their reciprocal influence, nor in which measure their relative combination contributes to the experience as a whole.

\section{Computational beauty}\label{sec:comp-beauty}
We observed how neuroaesthetics, although with some limitations, can provide us with useful information regarding our neurological reactions when we witness acts of beauty. If it is true that specific brain activity is observed in these situations, not so much we can say about whether these activities are caused by specific properties of the artistic -- specifically musical -- object. Research in computer science and artificial intelligence (\acsfont{AI}) has produced some (more or less valuable) results and theories, in some cases drawing from neuroaesthetics itself.

Once again, the domain of the visual arts has been the one where studies have been the most prolific. In fact, results show how several objective key properties seem to be present in beautiful images. \cite{jacobs2016aesthetics} observed that some of these properties correspond to lower spatial frequencies, oblique orientations, higher intensity variation, higher saturation, and overall redness.

\cite{schifanella2015image} developed a model which was able to surface beautiful but unpopular pictures from a pool of items uploaded to the photo-sharing platform Flickr. Their approach is based on computing specific descriptors related either to color (e.g., contrast, hue, saturation), spatial arrangement (e.g., symmetry, rule of thirds), or texture (e.g., entropy, energy, homogeneity), and comparing them against the same features computed from a ground-truth of crowdsourced pictures previously labelled as beautiful. As in the case of \cite{kawabata2004neural} mentioned in \autoref{sec:neuroaesthetics}, here the meaning of the term ``beautiful'' is not defined a priori; it was left to the users' own interpretation. Therefore, by not giving an explicit definition of beauty, we run the risk of including in the aesthetic judgment process a wide variety of criteria (such as preference, stylistic familiarity, popularity, memory, sympathy, elation\ldots) whose contribution to aesthetic experiences has not been fully explained yet.

Some theories that try to quantify beauty in music, or at least to give some related measure, have already been proposed. \cite{manaris2002progress}, and \cite{manaris2005zipf}, for example, conducted experiments exploiting a statistical technique known as Zipf’s law\footnote{Zipf's law is an empirical law formulated using mathematical statistics that refers to the fact that many types of data studied in the physical and social sciences can be approximated with a Zipfian distribution, where the most frequent class of datapoints will occur approximately twice as often as the second most frequent class, three times as often as the third most frequent class, etc. In the mentioned studies, this has been applied to many musical parameters (such as pitch, duration, melodic intervals, and harmonic consonance).} on a corpus of \acsfont{MIDI}-encoded pieces, suggesting that this technique might be used as a metric for aesthetic evaluation. The music pieces used in their experiments were reportedly selected <<by a member [...] with an extensive music theory background>>, are all pieces belonging to the classical music genre (as much as the vagueness of this label implies), and have been cut down to two minutes, to prevent fatigue in the listeners. These choices, for which no justification has been provided, could however introduce a strong bias to the experiment, since many assumptions are implicitly made here, or not explicitly discarded. One such bias is the fact that the music pieces have been chosen by just one person, with the only criteria that he has some knowledge in music theory.

\cite{hudson2011musical} advances an hypothesis that roots in information theory, proposing that compressibility and music appreciation are strictly bound. More specifically, the cognitive process of finding patterns more or less hidden inside a piece of music directly relates to a reward system responsible for our appreciation of it. This hypothesis, although fascinating, lacks the support of empirical experiments, and should therefore be taken with a grain of salt. A related study by \cite{mcdermott2013summary} shows that the auditory system tends to summarize temporal details of sound textures using time-averaged statistics, especially when the length of the sound is moderate to high.

\cite{brattico2017global}, on the same line, and drawing from the studies in visual aesthetics, put forward the hypothesis that our auditory system extracts global features from musical stimuli, and then passes them to the high-level processing responsible for the outcomes of the musical experience, including aesthetic judgment. These global features, analogously to visual features, are defined in terms of distribution of spectral energy, musical texture, expressivity, tempo and mode, and more. Moreover, they propose that the creation of musical beauty is not limited to any particular style, method, genre, or form, implying that the aforementioned model could be applied to any piece of music.

\section{The research question}\label{sec:question}
In the previous sections, I have briefly outlined some theories and approaches about beauty and aesthetic judgments. In the discussion I explained some of the many points of view presented from the perspective of a multitude of disciplines. By now, I hope the reader became aware of how incredibly complex and faceted the topic is, and how anyone willing to tame the problem even from a computational point of view should always at least provide the context they intend to work in, as many variables -- such as the methodology or interpretabilty of the results -- can be affected by these choices.

The apparent impossibility to find a way out from this labyrinth of opinions, studies, hypotheses should not discourage us to stop investigating; I rather see it as an indicator of the relevance of the problem as well as of the ongoing discussion  around it. People, regardless of what sayings tell us, \emph{do} argue over art, over music, over their own preferences, over beauty. Not only that: for the practical purposes of buying paintings and sculptures, judging flower competitions, awarding fashion prizes, granting scholarships, people \emph{need} to argue. We want to award the prize to the most beautiful roses, we want to choose the most beautiful painting submitted in the competition, we want to buy the most beautiful recording of a piece of music, and so on and so forth. There are critics who make a
living discussing the relative merits of films, musical compositions, concert
performances, paintings, plays and novels. The analysis of \emph{how} people argue over art is a task which I feel deserves more research efforts, especially given the impressive advancements in \acsfont{AI} and natural language processing techniques. In the present work, we have to draw some limits: we want to limit the scope of this research to music\footnote{Someone once said: <<Writing about music is like dancing about architecture>>; only God knows how much I disagree with that. Robert \citeauthor{christgau2005writing} gives a nice witty answer to those who so affirm: 
\begin{quote}
<<One of the many foolish things about the fools who compare writing about music to dancing about architecture is that dancing usually is about architecture. When bodies move in relation to a designed space, be it stage or ballroom or living room or gymnasium or agora or Congo Square, they comment on that space whether they mean to or not.>>

(\cite{christgau2005writing})
\end{quote}} and, of course, to beauty.

At the end of \autoref{subsec:subj-obj-inter} I hinted at \citeauthor{graham2005philosophy}'s reasons against subjectivism. He argues the following:
\begin{quote}
<<[...] In adducing reasons for my preference for a work of art (as for any object over which rational judgement ranges), there is at least one constraint that I am rationally obliged to acknowledge, the need to refer to features that the work actually possesses. I cannot plausibly say that I do not like \emph{The Waste Land} because I do not like limericks, for the obvious reason that \emph{The Waste Land} is not a limerick; I cannot give it as my reason for liking pre-Raphaelite painting that I prefer abstract to representational art, since pre-Raphaelite painting is as far from abstract art as one can get; I cannot justify my distaste for modernist architecture in terms of a more general dislike of excessive ornamentation, because famously modernist architecture eschews ornamentation; and so on.>>

(\cite{graham2005philosophy} -- Chapter 11)
\end{quote}

What Graham is telling us is that any aesthetic judgment must be carried out according to the actual features of the work about which it is a judgment. Otherwise, we would be talking about matters of mere preference, or personal taste. In other words, expressing an aesthetic judgment (i.e., saying that something is beautiful or ugly) is fundamentally different from statements such as <<I like the taste of avocado>> -- what in Hume's language could be defined as an \emph{original existence}: something that can be acknowledged, but about which not much more can be said. Furthermore, if calling something \emph{beautiful} was equivalent to expressing a simple preference, then why not simply doing so? When I say <<This is a \emph{beautiful} piece of music>>, why would I bother using a term in such a misleading objectified form, as if it was about the piece of music itself, when in fact it is only about me and my feelings towards it?

To wrap up, the points that will be taken for granted from now on, for the reasons discussed in this chapter, are:
\begin{enumerate}
	\item there is no agreement over the nature of beauty;
	\item because of this, it is hard to provide a unique definition of beauty;
	\item however, people talk about beauty;
	\item when expressing an aesthetic judgment, it is advisable to relate it to real properties of the object of the judgment;
	\item the act of giving an aesthetic judgment seems to imply the attribution of both \emph{(a)} a positive or negative value to the object, and \emph{(b)} an objectified status to the judgment itself.
\end{enumerate}

If we hold true these assumptions, and restricting our scope to music, I question whether there exist concepts that people tend to refer to when talking about musical beauty -- the ``real properties'' mentioned in point 4 of the previous list -- and, if so, whether it is possible to obtain them in a direct, automatic way starting from unstructured text sources, be they music reviews, comments about songs, playlists descriptions, etc. Thanks to the Internet, there are huge amounts of this kind of data we can take advantage of, while the field of natural language processing (\acsfont{NLP})\footnote{Natural Language Processing is a field of computer science and artificial intelligence that studies how to program computers to process and analyze large amounts of human natural language data.} offers us powerful techniques to extract information from such unstructured data.

I believe that incorporating an analysis of the proposed type into the already existing and ongoing research in philosophy, neurosci\-ences, and computer science can contribute with valuable insights over real case scenarios, insights that would otherwise need to be harvested over more conventional (and with less broad scope, although maybe more controlled) mediums, such as surveys or interviews.

\subsection{Limitations}\label{subsec:limits}
There are at least two dimensions in aesthetic judgments that have not been mentioned yet whose contribution must be held in mind, which I will here refer to as the \emph{dimensions of variability} of aesthetic experiences.

The first dimension of variability has to do with the observation that the majority of the studies presented here find their context within the boundaries of a Western tradition. The existence of differences between Eastern and Western aesthetics is a generally accepted notion, due to the fact that in non-Western societies aesthetics are more closely related to the communication of spiritual, ethical and philosophical meaning than in the Western tradition (\cite{anderson1989comparative}).

The second dimension lies in the temporal variable. Aesthetic experience varies throughout historical periods (\cite{pearce2016neuroaesthetics}), as cultural conventions have shifted or expanded. There are countless examples of artworks which
were popular in their day, but whose reputation has since fallen into obscurity, as well as there are examples of artworks which, on the other hand, have caused outrage  as soon as they were unveiled in all of their unconventional nature, but have since become admired staples of the repertoire (Igor Stravinsky's \emph{Le Sacre du Printemps} comes off the top of my head).

Therefore, the cultural and historical constitution of the concept of aesthetic experiences should be acknowledged. The choice of our data sources, as we will see in the next chapter, will be subject to these two limitations, as will be the generalizability of the results.

%*****************************************
%*****************************************
%*****************************************
%*****************************************
%*****************************************